% A Sample Thesis for the University of Calgary
% =============================================

% Load the UCalgary Memoir Thesis class. By default (without options),
% this produces a 1-1/2 spaced thesis in 11 point font without running
% heads. Use package options singlespaced or doublespaced for single
% or double line spacing. The default font is LaTeX's Computer
% Modern. Use package options times, palatino, libertine, garamond, or
% utopia for other nice typefaces. (You may need to install the
% relevant packages to get these to work.) Use package option headers
% if you want running heads. Use option fullpage if you want the text
% to occupy all allowable space (1 inch margins all around), or option
% manuscript if you want a page layout suitable for reviewing and
% proof reading. Option manuscript selects 12 pt type, doublespaced,
% approx. 25 lines per page, with approx. 72 characters per line. You
% may want to remove the option for the version you file in the Vault;
% it looks nicer and is a bit more compact. You can also use any
% options that the memoir class recognizes, such as 10pt for 10 point
% font, 11pt for 11 point font, 12pt for 12 point type,
% etc. Documentation of the memoir package can be found at
% https://ctan.org/pkg/memoir?lang=en

\documentclass[utopia,headers,manuscript]{ucalgmthesis}

% Using LaTeX? Then you're probably using math, and so you want to use
% the AMS math commands and define some theorem environments! But you
% can take these out or use your own favorite theorem package.

\usepackage{amsmath,amsthm}

\newtheorem{thm}{Theorem}
\theoremstyle{definition}
\newtheorem{defn}[thm]{Definition} % please number all of them together!

% microtype makes everything look better

\usepackage{microtype}

% We'll need some colored links, so we load xcolor and hyperref. But
% you can take that out if you don't want links at all or are happy
% with the standard garish colored boxes.

\usepackage[dvipsnames]{xcolor}

% You can turn off the boxes around links made by hyperref. Then links
% will appear in a different color, and per guidelines, all links must
% be blue or black. For blue links say

\usepackage[colorlinks,allcolors=MidnightBlue]{hyperref} 

% For black links, 
% \usepackage[hidelinks]{hyperref}

% If you prefer hyperref's boxes around links (which don't print), you
% can also change their color. With boxes around links, you probably
% don't want everything in the table of contents to be a link, so we
% only make the page numbers links.
%
% \usepackage[allbordercolors=Periwinkle,linktocpage]{hyperref}

% The table of contents in your PDF reader's sidebar is just titles by
% default, but it's nice to also have chapter and section numbers for
% easy navigation.

\usepackage[numbered]{bookmark}

% For author-year references, you probably want to use natbib with a
% bibliography style appropriate for your discipline; or check out
% latexbib!

\usepackage[round]{natbib}
\bibliographystyle{plainnat}

% The blindtext package produces the ``lorem ipsum''
% texts in this sample and can safely be removed.

\usepackage{blindtext}

% Now we put in the information for the thesis title page.

% Full Name

\author{Jane Mary Doe}

% Full Title

\title{An Important Contribution to the Literature}

% Official name of the degree

\degree{Doctor of Philosophy}

% The name of the graduate program (not the department!)

\prog{Graduate Program in Philosophy}

% The month (for the final version: when you file, not when you defended)

\monthname{May}

% The year

\thesisyear{2018}

% Tell hyperref to put author and title into the PDF metadata

\hypersetup{pdfinfo={Title={\thetitle},Author={\theauthor}}}

% If you want memoir to produce endnotes, turn them on here

\makepagenote

% Often you only want to output a single chapter so you can send it to
% your supervisor. Use includeonly and make sure everything you don't
% always want compiled to PDF is include'd from a separate file. For
% instance, to produce a PDF only of chapter 1, endnotes and
% bibliography, say

% \includeonly{chapter1,backmatter}

% To compile only the title page, which you need when submitting your
% thesis, say

% \includeonly{titlepage}

\begin{document}

\frontmatter

% titlepage.tex just makes the titlepage; it's in its own file so you
% can typeset it alone using includeonly.

% Sample University of Calgary Thesis 

% This file contains just the command to make the TITLE PAGE. It uses
% settings in the preamble of the main file.

\makethesistitle


% frontmatter.tex contains the abstract, preface, acknowledgments, and
% the commands to produce the table of contents, list of tables, etc.

\include{frontmatter}

% The main matter of the thesis contains the actual content, separated
% into chapters.

\mainmatter

\include{chapter1}

\include{chapter2}

% The back matter includes commands to produce endnotes, index,
% glossary, bibiliography, and the like.

\backmatter

\include{backmatter}

% The appendix contains material that would clutter up the main
% text. Remove it if you don't have an appendix.

\appendix

% Sample University of Calgary Thesis
% This file contains the APPENDIX

% If there is just one appendix, it must be called ``Appendix.'' For
% multiple appendices, use \chapter and a descriptive title,
% e.g., \chapter{Questionnaires}

\chapter*{Appendix}\label{appendix}

% \chapter* doesn't include it in the TOC, so we have to do that by
% hand.  If you have multiple appendices, use \chapter instead and
% remove the following line. 

\addcontentsline{toc}{chapter}{Appendix}

An appendix is a way to include important information that would
otherwise clutter up your thesis. It should be included when there is
additional relevant information that won't fit in the body of your
thesis. Any Appendix must also be mentioned in the body of your thesis
(e.g., ``For a full list of interview questions used, please see the
\hyperref[appendix]{Appendix}''). If your thesis only has one appendix, it
must be titled ``Appendix.'' If your thesis has more than one appendix,
add alphabetized letters, starting with ``Appendix~A.'' The following
are examples of things you might include in appendices:
\begin{enumerate}
  \item Copyright permissions with signatures removed
  \item Additional details of methodology and/or data
  \item Diagrams of equipment that you developed
  \item Digital files and/or artwork digital models
  \item Blank copies of questionnaires or surveys used in your research
\end{enumerate}


\end{document}
